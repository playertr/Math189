\documentclass[12pt,letterpaper]{hmcpset}
\usepackage[margin=1in]{geometry}
\usepackage{graphicx}
\usepackage{amsthm}
\usepackage{enumitem}
\usepackage{amsmath}

% Theorems
\usepackage{amsthm}
\renewcommand\qedsymbol{$\blacksquare$}
\makeatletter
\@ifclassloaded{article}{
    \newtheorem{definition}{Definition}[section]
    \newtheorem{example}{Example}[section]
    \newtheorem{theorem}{Theorem}[section]
    \newtheorem{corollary}{Corollary}[theorem]
    \newtheorem{lemma}{Lemma}[theorem]
}{
}
\makeatother

% Random Stuff
\setlength\unitlength{1mm}

\newcommand{\insertfig}[3]{
\begin{figure}[htbp]\begin{center}\begin{picture}(120,90)
\put(0,-5){\includegraphics[width=12cm,height=9cm,clip=]{#1.eps}}\end{picture}\end{center}
\caption{#2}\label{#3}\end{figure}}

\newcommand{\insertxfig}[4]{
\begin{figure}[htbp]
\begin{center}
\leavevmode \centerline{\resizebox{#4\textwidth}{!}{\input
#1.pstex_t}}
\caption{#2} \label{#3}
\end{center}
\end{figure}}

\long\def\comment#1{}

\newcommand\norm[1]{\left\lVert#1\right\rVert}
\DeclareMathOperator*{\argmin}{arg\,min}
\DeclareMathOperator*{\argmax}{arg\,max}

% bb font symbols
\newfont{\bbb}{msbm10 scaled 700}
\newcommand{\CCC}{\mbox{\bbb C}}

\newfont{\bbf}{msbm10 scaled 1100}
\newcommand{\CC}{\mbox{\bbf C}}
\newcommand{\PP}{\mbox{\bbf P}}
\newcommand{\RR}{\mbox{\bbf R}}
\newcommand{\QQ}{\mbox{\bbf Q}}
\newcommand{\ZZ}{\mbox{\bbf Z}}
\renewcommand{\SS}{\mbox{\bbf S}}
\newcommand{\FF}{\mbox{\bbf F}}
\newcommand{\GG}{\mbox{\bbf G}}
\newcommand{\EE}{\mbox{\bbf E}}
\newcommand{\NN}{\mbox{\bbf N}}
\newcommand{\KK}{\mbox{\bbf K}}
\newcommand{\KL}{\mbox{\bbf KL}}

% Vectors
\renewcommand{\aa}{{\bf a}}
\newcommand{\bb}{{\bf b}}
\newcommand{\cc}{{\bf c}}
\newcommand{\dd}{{\bf d}}
\newcommand{\ee}{{\bf e}}
\newcommand{\ff}{{\bf f}}
\renewcommand{\gg}{{\bf g}}
\newcommand{\hh}{{\bf h}}
\newcommand{\ii}{{\bf i}}
\newcommand{\jj}{{\bf j}}
\newcommand{\kk}{{\bf k}}
\renewcommand{\ll}{{\bf l}}
\newcommand{\mm}{{\bf m}}
\newcommand{\nn}{{\bf n}}
\newcommand{\oo}{{\bf o}}
\newcommand{\pp}{{\bf p}}
\newcommand{\qq}{{\bf q}}
\newcommand{\rr}{{\bf r}}
\renewcommand{\ss}{{\bf s}}
\renewcommand{\tt}{{\bf t}}
\newcommand{\uu}{{\bf u}}
\newcommand{\ww}{{\bf w}}
\newcommand{\vv}{{\bf v}}
\newcommand{\xx}{{\bf x}}
\newcommand{\yy}{{\bf y}}
\newcommand{\zz}{{\bf z}}
\newcommand{\0}{{\bf 0}}
\newcommand{\1}{{\bf 1}}

% Matrices
\newcommand{\Ab}{{\bf A}}
\newcommand{\Bb}{{\bf B}}
\newcommand{\Cb}{{\bf C}}
\newcommand{\Db}{{\bf D}}
\newcommand{\Eb}{{\bf E}}
\newcommand{\Fb}{{\bf F}}
\newcommand{\Gb}{{\bf G}}
\newcommand{\Hb}{{\bf H}}
\newcommand{\Ib}{{\bf I}}
\newcommand{\Jb}{{\bf J}}
\newcommand{\Kb}{{\bf K}}
\newcommand{\Lb}{{\bf L}}
\newcommand{\Mb}{{\bf M}}
\newcommand{\Nb}{{\bf N}}
\newcommand{\Ob}{{\bf O}}
\newcommand{\Pb}{{\bf P}}
\newcommand{\Qb}{{\bf Q}}
\newcommand{\Rb}{{\bf R}}
\newcommand{\Sb}{{\bf S}}
\newcommand{\Tb}{{\bf T}}
\newcommand{\Ub}{{\bf U}}
\newcommand{\Wb}{{\bf W}}
\newcommand{\Vb}{{\bf V}}
\newcommand{\Xb}{{\bf X}}
\newcommand{\Yb}{{\bf Y}}
\newcommand{\Zb}{{\bf Z}}

% Calligraphic
\newcommand{\Ac}{{\cal A}}
\newcommand{\Bc}{{\cal B}}
\newcommand{\Cc}{{\cal C}}
\newcommand{\Dc}{{\cal D}}
\newcommand{\Ec}{{\cal E}}
\newcommand{\Fc}{{\cal F}}
\newcommand{\Gc}{{\cal G}}
\newcommand{\Hc}{{\cal H}}
\newcommand{\Ic}{{\cal I}}
\newcommand{\Jc}{{\cal J}}
\newcommand{\Kc}{{\cal K}}
\newcommand{\Lc}{{\cal L}}
\newcommand{\Mc}{{\cal M}}
\newcommand{\Nc}{{\cal N}}
\newcommand{\Oc}{{\cal O}}
\newcommand{\Pc}{{\cal P}}
\newcommand{\Qc}{{\cal Q}}
\newcommand{\Rc}{{\cal R}}
\newcommand{\Sc}{{\cal S}}
\newcommand{\Tc}{{\cal T}}
\newcommand{\Uc}{{\cal U}}
\newcommand{\Wc}{{\cal W}}
\newcommand{\Vc}{{\cal V}}
\newcommand{\Xc}{{\cal X}}
\newcommand{\Yc}{{\cal Y}}
\newcommand{\Zc}{{\cal Z}}

% Bold greek letters
\newcommand{\alphab}{\hbox{\boldmath$\alpha$}}
\newcommand{\betab}{\hbox{\boldmath$\beta$}}
\newcommand{\gammab}{\hbox{\boldmath$\gamma$}}
\newcommand{\deltab}{\hbox{\boldmath$\delta$}}
\newcommand{\etab}{\hbox{\boldmath$\eta$}}
\newcommand{\lambdab}{\hbox{\boldmath$\lambda$}}
\newcommand{\epsilonb}{\hbox{\boldmath$\epsilon$}}
\newcommand{\nub}{\hbox{\boldmath$\nu$}}
\newcommand{\mub}{\hbox{\boldmath$\mu$}}
\newcommand{\zetab}{\hbox{\boldmath$\zeta$}}
\newcommand{\phib}{\hbox{\boldmath$\phi$}}
\newcommand{\psib}{\hbox{\boldmath$\psi$}}
\newcommand{\thetab}{\hbox{\boldmath$\theta$}}
\newcommand{\taub}{\hbox{\boldmath$\tau$}}
\newcommand{\omegab}{\hbox{\boldmath$\omega$}}
\newcommand{\xib}{\hbox{\boldmath$\xi$}}
\newcommand{\sigmab}{\hbox{\boldmath$\sigma$}}
\newcommand{\pib}{\hbox{\boldmath$\pi$}}
\newcommand{\rhob}{\hbox{\boldmath$\rho$}}

\newcommand{\Gammab}{\hbox{\boldmath$\Gamma$}}
\newcommand{\Lambdab}{\hbox{\boldmath$\Lambda$}}
\newcommand{\Deltab}{\hbox{\boldmath$\Delta$}}
\newcommand{\Sigmab}{\hbox{\boldmath$\Sigma$}}
\newcommand{\Phib}{\hbox{\boldmath$\Phi$}}
\newcommand{\Pib}{\hbox{\boldmath$\Pi$}}
\newcommand{\Psib}{\hbox{\boldmath$\Psi$}}
\newcommand{\Thetab}{\hbox{\boldmath$\Theta$}}
\newcommand{\Omegab}{\hbox{\boldmath$\Omega$}}
\newcommand{\Xib}{\hbox{\boldmath$\Xi$}}

% mixed symbols
\newcommand{\sinc}{{\hbox{sinc}}}
\newcommand{\diag}{{\hbox{diag}}}
\renewcommand{\det}{{\hbox{det}}}
\newcommand{\trace}{{\hbox{tr}}}
\newcommand{\tr}{\trace}
\newcommand{\sign}{{\hbox{sign}}}
\renewcommand{\arg}{{\hbox{arg}}}
\newcommand{\var}{{\hbox{var}}}
\newcommand{\cov}{{\hbox{cov}}}
\renewcommand{\Re}{{\rm Re}}
\renewcommand{\Im}{{\rm Im}}
\newcommand{\eqdef}{\stackrel{\Delta}{=}}
\newcommand{\defines}{{\,\,\stackrel{\scriptscriptstyle \bigtriangleup}{=}\,\,}}
\newcommand{\<}{\left\langle}
\renewcommand{\>}{\right\rangle}
\newcommand{\Psf}{{\sf P}}
\newcommand{\T}{\top}
\newcommand{\m}[1]{\begin{bmatrix} #1 \end{bmatrix}}


% info for header block in upper right hand corner
\name{Tim Player}
\class{Math189R SU17}
\assignment{Homework 1}
\duedate{Tuesday, May 15, 2018}

\renewcommand{\labelenumi}{{(\alph{enumi})}}


\begin{document}

\begin{problem}[1]
(\textbf{Linear Transformation}) Let $\mathbf{y} = A\mathbf{x} + \mathbf{b}$ be a random vector.
show that expectation is linear:
\[
    \EE[\yy] = \EE[A\xx + \bb] = A\EE[\xx] + \bb.
\]
Also show that
\[
    \cov[\yy] = \cov[A\xx + \bb] = A \cov[\xx] A^\T = A\Sigmab A^\T.
\]
\end{problem}
\begin{solution}
\begin{enumerate}

\item
In this solution, we assume $A$ and $\mathbf{b}$ are a constant matrix and vector, respectively, with all the randomness being introduced by $\mathbf{x}$.

Then, let $A$ by a square $m \times m $ matrix with of the form

\[
A = \begin{bmatrix} 
    a_{11} & \dots & a_{1m} \\
    \vdots & \ddots & \\
    a_{m1} &        & a_{mm} 
    \end{bmatrix}
\]

and let $\mathbf{b}$ be an $m$-dimension column vector
    
   \[
b=
  \begin{bmatrix}
    b_{1} \\
    \vdots \\
    b_{n} \\
  \end{bmatrix}.
\]

We can assume without loss of generality that $\mathbf{y}$ is a discrete random variable such that
 
\[
\EE[\yy] = \frac{1}{n} \sum_{i=1}^{n} y_i .
\]

Then,
\begin{align}
\EE[\yy] &= \EE[A\xx + \bb]  \\
&= \EE[A\xx]  + \EE[\bb] \\
\intertext{since summations can be distributed to additive terms. Trivially, since, $\bb$ is a constant vector, $\EE[\bb] = \bb$ so}
\EE[\yy] &= \EE[A\xx]  + \bb. \\
\intertext{Then, the matrix multiplication may be distributed so that}
\EE[\yy] &= \EE   \  
	\begin{bmatrix}
    		a_{11} \, x_1  + \dots + a_{1n} \, x_n  \\
    		\vdots \\
    		a_{m1} \, x_1  + \dots + a_{mn} \, x_n \\
  	\end{bmatrix}
	+ \bb. \\
\intertext{Since the expected value of a vector is simply a vector of each component's expected value, then}
\EE[\yy] &=  
	\begin{bmatrix}
    		\EE[a_{11} \, x_1  + \dots + a_{1n} \, x_n ] \\
    		\vdots \\
    		\EE[a_{m1} \, x_1  + \dots + a_{mn} \, x_n ] \\
  	\end{bmatrix}
	+ \bb. \\
\intertext{Because the terms of $A$ are constant, they may be "pulled out" of the multiplication, leaving}
\EE[\yy] &=  A \begin{bmatrix}
    		\EE[x_1] \\
    		\vdots \\
    		\EE[x_n] \\
  	\end{bmatrix}
	+ \bb. \\
&=A \EE[\xx] + \bb
\end{align}
as desired, showing that expectation is linear.

\item
We use the linear nature of expected value to prove that
\[
    \cov[\yy] = \cov[A\xx + \bb] = A \cov[\xx] A^\T = A\Sigmab A^\T.
\]

First, we preface that, from Wikipedia, the covariance of a matrix $\mathbf{X}$ is 
\[
\Sigmab = \EE[ (\mathbf{X} - \EE[\mathbf{X}])(\mathbf{X} - \EE[\mathbf{X}])^\T ].
\]

Thus, we may write
\begin{align*}
	\cov[\yy] &= \cov[A\xx + \bb] \\
	&= \EE [ (A\xx + \bb - \EE[A\xx + \bb ])(A\xx + \bb  - \EE[A\xx + \bb )^\T ].
\end{align*}
We then apply expectation as a linear operator to state
\begin{align*}
	\cov[\yy] &= \EE [ (A\xx + \bb - \EE[A\xx]  - \EE[\bb ])(A\xx + \bb  - \EE[A\xx] - \EE[\bb] )^\T ]\\
	&= \EE [ (A\xx + \bb - \EE[A\xx]  - \bb)(A\xx + \bb  - \EE[A\xx] - \bb )^\T ] \\
	&= \EE [ (A\xx - \EE[A\xx] )(A\xx  - \EE[A\xx] )^\T ]\\
	&= \EE [ (A\xx - A\EE[\xx] )(A\xx  - A\EE[\xx] )^\T ]\\
	&= \EE [ (A (\xx - \EE[\xx]) )(A(\xx  - \EE[\xx]) )^\T ]\\
\end{align*}

Since in general, matrix multiplication transposes work so $(JK)^\T = K^\T J^\T$, we may write

\begin{align*}
	\cov[\yy] &= \EE [ A (\xx - \EE[\xx])(\xx  - \EE[\xx])^\T A^\T) ]\\
\end{align*}

And, by associativity and the definition of $\Sigmab$, 

\begin{align*}
	\cov[\yy] = A\Sigmab A^\T,
\end{align*}
as desired.

\end{enumerate}
    \vfill
\end{solution}
\newpage




\begin{problem}[2]
Given the dataset $\Dc = \{(x,y)\} = \{(0,1), (2,3), (3,6), (4,8)\}$
\begin{enumerate}
   \item Find the least squares estimate $y = \thetab^\T\xx$ by hand using
        Cramer's Rule.
    \item Use the normal equations to find the same solution and verify it
        is the same as part (a).
    \item Plot the data and the optimal linear fit you found.
    \item Find randomly generate 100 points near the line with white Gaussian
        noise and then compute the least squares estimate (using a computer).
        Verify that this new line is close to the original and plot the new
        dataset, the old line, and the new line.
\end{enumerate}

\end{problem}
\begin{solution}
\begin{enumerate}

\item
In lecture, we found using Cramer's rule that a set of data $\Dc = \{(x,y)\}$ can be fitted with a line of slope 
\[
m = \frac{ n \sum_{i=1}^n x_i y_i - (\sum_{i=1}^n x_i)(\sum_{i=1}^n y_i)}
	{n \sum_{i=1}^n x_i^2 - (\sum_{i=1}^n x_i)^2}
\]
and y-intercept
\[
b = \frac{  (\sum_{i=1}^n x_i^2)(\sum_{i=1}^n y_i) - (\sum_{i=1}^n x_i)(\sum_{i=1}^n x_i y_i) }
	{n \sum_{i=1}^n x_i^2 - (\sum_{i=1}^n x_i)^2}.
\]
Such a line will minimize the sum of the squared residuals.

By hand, we thus first tally the relevant summations:
\begin{align*}
\sum_{i=1}^n x_i &=  0 + 2 + 3 + 4 &= 9\\
\sum_{i=1}^n y_i &= 1 + 3 + 6 + 8 &= 18\\
\sum_{i=1}^n x_i y_i &= 0 \times 1 + 2 \times 3  + 3 \times 6 + 4 \times 8 &= 56 \\
\sum_{i=1}^n x_i^2 &= 0^2 + 2^2 + 3^2 + 4^2 &= 29.\\
\end{align*}
We then insert the totals into the formulae for $m$ and $b$ to yield our answer.
\begin{align*}
m &= \frac{4 (56) - (9)(18)}{4(29) - (9)^2}    &= \frac{62}{35} &\approx 1.7714 \\
b &= \frac{ (29)(18) - (9)(56) }{4(29) - (9)^2} &= \frac{18}{35} &\approx 0.5143
\end{align*}

\item
In lecture, we derived the normal equation.
Let the design matrix be the matrix of inputs. We prepend a column of ones to serve as the constant offset, so
\[
\mathbf{X} = 
	\begin{bmatrix} 
    		1 & 0  \\
    		1 & 2  \\
		1 & 3  \\
    		1 & 4 
    \end{bmatrix}.
\]
The normal equation states that the fit parameters $\thetab$ minimize the sum of the squared error when
\[
\Xb^\T \yy - \Xb^\T \Xb \thetab = 0.\\
\]
Since  $(\Xb^\T \Xb)$ is invertible, 
\[
\thetab = (\Xb^\T \Xb)^{-1} (\Xb^\T \yy).
\]

We may easily find the matrix $(\Xb^\T \Xb)^{-1}$ with linear algebra.

\begin{align*}
\Xb^\T \Xb &= \begin{bmatrix} 
  		1 & 1 & 1 & 1 \\
		0 & 2 & 3 & 4 \\
    	      \end{bmatrix}
	      \begin{bmatrix} 
    		1 & 0  \\
    		1 & 2  \\
		1 & 3  \\
    		1 & 4 
    	      \end{bmatrix}
	      =\begin{bmatrix} 
    		4 & 9  \\
    		9 & 29  \\
    	      \end{bmatrix} \\
(\Xb^\T \Xb)^{-1} &=  \begin{bmatrix} 
    		29/35 & -9/35  \\
    		-9/35 & 4/35 \\
    	      \end{bmatrix}
\end{align*}

Then,
\begin{align*}
\thetab &= (\Xb^\T \Xb)^{-1} (\Xb^\T \yy) \\
&= 	\begin{bmatrix} 
    		29/35 & -9/35  \\
    		-9/35 & 4/35 \\
    	\end{bmatrix}
	\begin{bmatrix} 
  		1 & 1 & 1 & 1 \\
		0 & 2 & 3 & 4 \\
   	\end{bmatrix}
	\begin{bmatrix} 
    		1  \\
    		3  \\
		6 \\
    		8
   	\end{bmatrix} \\
&= 	\begin{bmatrix} 
    		29/35 & -9/35  \\
    		-9/35 & 4/35 \\
    	\end{bmatrix}
	\begin{bmatrix} 
		18\\
		56
   	\end{bmatrix} \\
&=	\begin{bmatrix} 
		18/35\\
		62/35
   	\end{bmatrix}
\end{align*}

The values of $b$ and $m$ match the ones derived with Cramer's rule.


\end{enumerate}
    \vfill
\end{solution}
\newpage



\end{document}

