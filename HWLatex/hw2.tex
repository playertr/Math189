\documentclass[12pt,letterpaper]{hmcpset}
\usepackage[margin=1in]{geometry}
\usepackage{graphicx}
\usepackage{amsmath,amssymb}
\usepackage{enumerate}
\usepackage{hyperref}
\usepackage{parskip}

% Theorems
\usepackage{amsthm}
\renewcommand\qedsymbol{$\blacksquare$}
\makeatletter
\@ifclassloaded{article}{
    \newtheorem{definition}{Definition}[section]
    \newtheorem{example}{Example}[section]
    \newtheorem{theorem}{Theorem}[section]
    \newtheorem{corollary}{Corollary}[theorem]
    \newtheorem{lemma}{Lemma}[theorem]
}{
}
\makeatother

% Random Stuff
\setlength\unitlength{1mm}

\newcommand{\insertfig}[3]{
\begin{figure}[htbp]\begin{center}\begin{picture}(120,90)
\put(0,-5){\includegraphics[width=12cm,height=9cm,clip=]{#1.eps}}\end{picture}\end{center}
\caption{#2}\label{#3}\end{figure}}

\newcommand{\insertxfig}[4]{
\begin{figure}[htbp]
\begin{center}
\leavevmode \centerline{\resizebox{#4\textwidth}{!}{\input
#1.pstex_t}}
\caption{#2} \label{#3}
\end{center}
\end{figure}}

\long\def\comment#1{}

\newcommand\norm[1]{\left\lVert#1\right\rVert}
\DeclareMathOperator*{\argmin}{arg\,min}
\DeclareMathOperator*{\argmax}{arg\,max}

% bb font symbols
\newfont{\bbb}{msbm10 scaled 700}
\newcommand{\CCC}{\mbox{\bbb C}}

\newfont{\bbf}{msbm10 scaled 1100}
\newcommand{\CC}{\mbox{\bbf C}}
\newcommand{\PP}{\mbox{\bbf P}}
\newcommand{\RR}{\mbox{\bbf R}}
\newcommand{\QQ}{\mbox{\bbf Q}}
\newcommand{\ZZ}{\mbox{\bbf Z}}
\renewcommand{\SS}{\mbox{\bbf S}}
\newcommand{\FF}{\mbox{\bbf F}}
\newcommand{\GG}{\mbox{\bbf G}}
\newcommand{\EE}{\mbox{\bbf E}}
\newcommand{\NN}{\mbox{\bbf N}}
\newcommand{\KK}{\mbox{\bbf K}}
\newcommand{\KL}{\mbox{\bbf KL}}

% Vectors
\renewcommand{\aa}{{\bf a}}
\newcommand{\bb}{{\bf b}}
\newcommand{\cc}{{\bf c}}
\newcommand{\dd}{{\bf d}}
\newcommand{\ee}{{\bf e}}
\newcommand{\ff}{{\bf f}}
\renewcommand{\gg}{{\bf g}}
\newcommand{\hh}{{\bf h}}
\newcommand{\ii}{{\bf i}}
\newcommand{\jj}{{\bf j}}
\newcommand{\kk}{{\bf k}}
\renewcommand{\ll}{{\bf l}}
\newcommand{\mm}{{\bf m}}
\newcommand{\nn}{{\bf n}}
\newcommand{\oo}{{\bf o}}
\newcommand{\pp}{{\bf p}}
\newcommand{\qq}{{\bf q}}
\newcommand{\rr}{{\bf r}}
\renewcommand{\ss}{{\bf s}}
\renewcommand{\tt}{{\bf t}}
\newcommand{\uu}{{\bf u}}
\newcommand{\ww}{{\bf w}}
\newcommand{\vv}{{\bf v}}
\newcommand{\xx}{{\bf x}}
\newcommand{\yy}{{\bf y}}
\newcommand{\zz}{{\bf z}}
\newcommand{\0}{{\bf 0}}
\newcommand{\1}{{\bf 1}}

% Matrices
\newcommand{\Ab}{{\bf A}}
\newcommand{\Bb}{{\bf B}}
\newcommand{\Cb}{{\bf C}}
\newcommand{\Db}{{\bf D}}
\newcommand{\Eb}{{\bf E}}
\newcommand{\Fb}{{\bf F}}
\newcommand{\Gb}{{\bf G}}
\newcommand{\Hb}{{\bf H}}
\newcommand{\Ib}{{\bf I}}
\newcommand{\Jb}{{\bf J}}
\newcommand{\Kb}{{\bf K}}
\newcommand{\Lb}{{\bf L}}
\newcommand{\Mb}{{\bf M}}
\newcommand{\Nb}{{\bf N}}
\newcommand{\Ob}{{\bf O}}
\newcommand{\Pb}{{\bf P}}
\newcommand{\Qb}{{\bf Q}}
\newcommand{\Rb}{{\bf R}}
\newcommand{\Sb}{{\bf S}}
\newcommand{\Tb}{{\bf T}}
\newcommand{\Ub}{{\bf U}}
\newcommand{\Wb}{{\bf W}}
\newcommand{\Vb}{{\bf V}}
\newcommand{\Xb}{{\bf X}}
\newcommand{\Yb}{{\bf Y}}
\newcommand{\Zb}{{\bf Z}}

% Calligraphic
\newcommand{\Ac}{{\cal A}}
\newcommand{\Bc}{{\cal B}}
\newcommand{\Cc}{{\cal C}}
\newcommand{\Dc}{{\cal D}}
\newcommand{\Ec}{{\cal E}}
\newcommand{\Fc}{{\cal F}}
\newcommand{\Gc}{{\cal G}}
\newcommand{\Hc}{{\cal H}}
\newcommand{\Ic}{{\cal I}}
\newcommand{\Jc}{{\cal J}}
\newcommand{\Kc}{{\cal K}}
\newcommand{\Lc}{{\cal L}}
\newcommand{\Mc}{{\cal M}}
\newcommand{\Nc}{{\cal N}}
\newcommand{\Oc}{{\cal O}}
\newcommand{\Pc}{{\cal P}}
\newcommand{\Qc}{{\cal Q}}
\newcommand{\Rc}{{\cal R}}
\newcommand{\Sc}{{\cal S}}
\newcommand{\Tc}{{\cal T}}
\newcommand{\Uc}{{\cal U}}
\newcommand{\Wc}{{\cal W}}
\newcommand{\Vc}{{\cal V}}
\newcommand{\Xc}{{\cal X}}
\newcommand{\Yc}{{\cal Y}}
\newcommand{\Zc}{{\cal Z}}

% Bold greek letters
\newcommand{\alphab}{\hbox{\boldmath$\alpha$}}
\newcommand{\betab}{\hbox{\boldmath$\beta$}}
\newcommand{\gammab}{\hbox{\boldmath$\gamma$}}
\newcommand{\deltab}{\hbox{\boldmath$\delta$}}
\newcommand{\etab}{\hbox{\boldmath$\eta$}}
\newcommand{\lambdab}{\hbox{\boldmath$\lambda$}}
\newcommand{\epsilonb}{\hbox{\boldmath$\epsilon$}}
\newcommand{\nub}{\hbox{\boldmath$\nu$}}
\newcommand{\mub}{\hbox{\boldmath$\mu$}}
\newcommand{\zetab}{\hbox{\boldmath$\zeta$}}
\newcommand{\phib}{\hbox{\boldmath$\phi$}}
\newcommand{\psib}{\hbox{\boldmath$\psi$}}
\newcommand{\thetab}{\hbox{\boldmath$\theta$}}
\newcommand{\taub}{\hbox{\boldmath$\tau$}}
\newcommand{\omegab}{\hbox{\boldmath$\omega$}}
\newcommand{\xib}{\hbox{\boldmath$\xi$}}
\newcommand{\sigmab}{\hbox{\boldmath$\sigma$}}
\newcommand{\pib}{\hbox{\boldmath$\pi$}}
\newcommand{\rhob}{\hbox{\boldmath$\rho$}}

\newcommand{\Gammab}{\hbox{\boldmath$\Gamma$}}
\newcommand{\Lambdab}{\hbox{\boldmath$\Lambda$}}
\newcommand{\Deltab}{\hbox{\boldmath$\Delta$}}
\newcommand{\Sigmab}{\hbox{\boldmath$\Sigma$}}
\newcommand{\Phib}{\hbox{\boldmath$\Phi$}}
\newcommand{\Pib}{\hbox{\boldmath$\Pi$}}
\newcommand{\Psib}{\hbox{\boldmath$\Psi$}}
\newcommand{\Thetab}{\hbox{\boldmath$\Theta$}}
\newcommand{\Omegab}{\hbox{\boldmath$\Omega$}}
\newcommand{\Xib}{\hbox{\boldmath$\Xi$}}

% mixed symbols
\newcommand{\sinc}{{\hbox{sinc}}}
\newcommand{\diag}{{\hbox{diag}}}
\renewcommand{\det}{{\hbox{det}}}
\newcommand{\trace}{{\hbox{tr}}}
\newcommand{\tr}{\trace}
\newcommand{\sign}{{\hbox{sign}}}
\renewcommand{\arg}{{\hbox{arg}}}
\newcommand{\var}{{\hbox{var}}}
\newcommand{\cov}{{\hbox{cov}}}
\renewcommand{\Re}{{\rm Re}}
\renewcommand{\Im}{{\rm Im}}
\newcommand{\eqdef}{\stackrel{\Delta}{=}}
\newcommand{\defines}{{\,\,\stackrel{\scriptscriptstyle \bigtriangleup}{=}\,\,}}
\newcommand{\<}{\left\langle}
\renewcommand{\>}{\right\rangle}
\newcommand{\Psf}{{\sf P}}
\newcommand{\T}{\top}
\newcommand{\m}[1]{\begin{bmatrix} #1 \end{bmatrix}}

% info for header block in upper right hand corner
\name{Tim Player}
\class{Math189R SU17}
\assignment{Homework 2}
\duedate{Wednesday, May 16, 2017}

\begin{document}

Feel free to work with other students, but make sure you write up the homework
and code on your own (no copying homework \textit{or} code; no pair programming).
Feel free to ask students or instructors for help debugging code or whatever else,
though.
\newline
\newline
The starter files can be found under the Resource tab on course website. The graphs for problem 3 generated by the sample solution could be found in the corresponding zipfile. These graphs only serve as references to your implementation. You should generate your own graphs for submission. Please print out all the graphs generated by your own code and submit them together with the written part, and make sure you upload the code to your Github repository.\\

\begin{problem}[1]
	(\textbf{Murphy 8.3}) Gradient and Hessian of the log-likelihood for
	logistic regression.
	\begin{enumerate}[(a)]
		\item Let $\sigma(x) = \frac{1}{1 + e^{-x}}$ be the sigmoid function. Show that
		\[
		\sigma'(x) = \sigma(x)\left[1 - \sigma(x)\right].
		\]
		\item Using the previous result and the chain rule of calculus, derive an
		expression for the gradient of the log likelihood for logistic regression.
		\item The Hessian can be written as $\Hb=\Xb^\T\Sb\Xb$ where $\Sb =
		\diag(\mu_1(1-\mu_1), \dots, \mu_n(1-\mu_n))$. Derive this and show that
		$\Hb \succeq 0$ ($A \succeq 0$ means that $A$ is positive semidefinite).\\
	\end{enumerate} 

\textit{Hint:} Use the \textbf{negative} log-likelihood of logistic regression for this problem.
\end{problem}
\begin{solution}

	\begin{enumerate}[(a)]
		\item 
		To verify the equation, we solve both sides independently and show them to 
		be equal.
		\begin{align*}
			\sigma'(x) &= \sigma(x)\left[1 - \sigma(x)\right] \\
			\frac{1}{e^x (1 + e^{-x})^2} &= 
				\frac{1}{1 + e^{-x}}[1 - \frac{1}{1 + e^{-x}}] \\
			\frac{1}{e^x (1 + e^{-x})^2} &= 
				\frac{1}{1 + e^{-x}}- \frac{1}{(1 + e^{-x})^2} \\
			\frac{1}{e^x (1 + e^{-x})^2} &= 
				\frac{1 + e^{-x} - 1}{(1 + e^{-x})^2} \\
			\frac{1}{e^x (1 + e^{-x})^2} &= 
				\frac{1}{e^x (1 + e^{-x})^2} 
		\end{align*}
		As desired, the two sides are equivalent. Hence,
		\[
			\sigma'(x) = \sigma(x)\left[1 - \sigma(x)\right].	
		\]

		\item
		In class, we found that the negative log likelihood for logistic regression is 
		\[
			nll(\thetab) = - \sum_i y^{(i)} \log( \sigma( \thetab^\T \xx^{(i)}) ) 
				+ (1 - y^{(i)}) \log( 1 - \sigma( \thetab^\T \xx^{(i)}) )
		\]
		Using the chain rule and the result above,
		\begin{align*}
			\nabla_{\thetab} nll(\thetab) &= - \sum_i y^{(i)} \frac{1}{\sigma( \thetab^\T \xx^{(i)})}
				\sigma'( \thetab^\T \xx^{(i)}) - (1 - y^{(i)}) 
				\frac{1}{1 - \sigma( \thetab^\T \xx^{(i)})}
				\sigma'( \thetab^\T \xx^{(i)}) \\
			&= - \sum_i y^{(i)} [1 - \sigma( \thetab^\T \xx^{(i)})] \xx^{(i)} - (1 	- y^{(i)}) \sigma( \thetab^\T \xx^{(i)}) \xx^{(i)} \\
			&= - \sum_i \xx^{(i)}  y^{(i)} - y^{(i)} \sigma( \thetab^\T \xx^{(i)}) \xx^{(i)} - \sigma( \thetab^\T \xx^{(i)}) \xx^{(i)}  + y^{(i)} \sigma( \thetab^\T \xx^{(i)}) \xx^{(i)} \\
			&= - \sum_i (y^{(i)} - \sigma( \thetab^\T \xx^{(i)})) \xx^{(i)} \\
			&= - \sum_i (y^{(i)} - \mu^{(i)}) \xx^{(i)} \\
			&= X^\T (\mub - \yy)
		\end{align*}
		where $y^{(i)}$ is the $i^{\text{th}}$ entry in the results column vector, $\mu^{(i)}$ is $\sigma( \thetab^\T \xx^{(i)})$, $X$ is the domain matrix, and $\xx^{(i)}$ is the transpose of the $i^{\text{th}}$ row in the design matrix (corresponding to the $i^{\text{th}}$ data observation).

		\item 
		The Hessian matrix of the NLL function is the gradient of the gradient in row form, i.e.
		\[
			\Hb_{\thetab} = \nabla_{\thetab}( \nabla_{\thetab} nll(\thetab))^\T
		\]
		Then, we may expand at continue.
		\begin{align*}
			\Hb_{\thetab} &=  \nabla_{\thetab}( X^\T (\mub - \yy) )^\T \\
			&= \nabla_{\thetab}( X^\T \mub - X^\T \yy) \\
			&= \nabla_{\thetab}( \mub^\T X - \yy^\T X) \\
			&= \nabla_{\thetab}( \mub^\T X ) \\
			&= \nabla_{\thetab}( \sigma( X \thetab)^\T X)
		\end{align*}

		We now note that $\sigma( X \thetab)^\T X$ is a row vector comprised of 
		\[
			[\xx_1 \cdot \sigma(\xx_1 \cdot \thetab), \, ... \, , \,  \xx_m \cdot \\sigma(\xx_1 \cdot \thetab)],
		\]
		where $\xx_i$ is the $i^{\text{th}}$ row of the domain matrix. Thus, it may be represented as 
		\[
			\sigma( X \thetab)^\T X = X^\T M,
		\]
		where M is a diagonal matrix whose entries are the ordered entries of $\sigma(X \thetab)$,
		\[
			M = 
			\begin{bmatrix} 
				\sigma(\xx_1 \cdot \thetab) & &  \\
				 & \ddots & \\
				 &  & \sigma(\xx_m \cdot \thetab)
			\end{bmatrix}.
		\]
		Then,
		 
		\begin{align*}
			\Hb_{\thetab} &= \nabla_{\thetab}( X^\T M )\\
			&= X^\T (\nabla_{\thetab}M)
			&= X^\T S X)
		\end{align*}
		Where $S = \diag(\mu_1(1 - \mu_1), \, ... \, \mu_m(1 - \mu_m)$).

		Trivially, $\Hb_{\thetab}$ is positive semi-definite because the diagonal entries of $S$, its eigenvalues, are nonnegative since they are the product of outputs of the logistic function and thus reside in $(0, 1)$.
		



	\end{enumerate}
	\vfill
\end{solution}
\newpage

\begin{problem}[2]
	(\textbf{Murphy 2.11})
	Derive the normalization constant ($Z$) for a one dimensional
	zero-mean Gaussian
	\[
	\PP(x; \sigma^2) = \frac{1}{Z}\exp\left(-\frac{x^2}{2\sigma^2}\right)
	\]
	such that $\PP(x; \sigma^2)$ becomes a valid density.
\end{problem}
\begin{solution}
	A valid probability density must integrate to 1 over $\mathbb{R}$.
	First, let's see what the integral over $\mathbb{R}$ is without a normalization constant. We will find the square of that integral $Z$.
	\begin{align*}
		Z^2 &= \int_a^b \int_a^b e^{(\frac{x^2 P y^2}{2 \sigma^2})} \, dx \, dy\\
		&= \int_0^{2\pi} \int_0^{\infty} r e^{(\frac{-r^2}{2 \sigma^2})} \, dr \, d\theta
	\end{align*}
	Let $u = e^{(\frac{-r^2}{2 \sigma^2}})$. Then $\frac{du}{dr} = -\frac{r}{\sigma^2} e^{(\frac{-r^2}{2 \sigma^2})}$.
	\begin{align*}
		Z^2 &= \int_0^{2\pi} \int_1^{0} \frac{r u \sigma^2}{r e^{(\frac{-r^2}{2 \sigma^2})}} \, du \, d\theta\\
		&= \int_0^{2\pi} \int_1^{0} -\sigma^2 \, du \, d\theta\\
		&= 2 \pi \sigma^2
	\end{align*}

	Because $Z = \sigma \sqrt{2 \pi}$, the normalization factor is $\frac{1}{Z} = \frac{1}{\sigma \sqrt{2 \pi}}$.

	\vfill
\end{solution}
\newpage

\begin{problem}[3]
(\textbf{regression}). In this problem, we will use the online news popularity dataset to set up a model for linear regression. In the starter code, we have already parsed the data for you. However, you might need internet connection to access the data and therefore successfully run the starter code.
\newline \newline
We split the csv file into a training and test set with
the first two thirds of the data in the training set and the rest for testing.
Of the testing data, we split the first half into a `validation set' (used
to optimize hyperparameters while leaving your testing data pristine) and
the remaining half as your test set.
We will use this data for the remainder of the problem. The goal of this data
is to predict the \textbf{log} number of shares a news article will have given the other
features.
\newline \newline
\begin{enumerate}[(a)]
	\item (\textbf{math}) Show that the maximum a posteriori problem for
	linear regression with a zero-mean Gaussian prior $\PP(\ww) = \prod_j
	\Nc(w_j | 0, \tau^2)$ on the weights,
	\[
	\argmax_\ww \sum_{i=1}^N \log\Nc(y_i | w_0 + \ww^\T\xx_i, \sigma^2) + \sum_{j=1}^D \log\Nc(w_j | 0, \tau^2)
	\]
	is equivalent to the ridge regression problem
	\[
	\argmin \frac{1}{N}\sum_{i=1}^N (y_i - (w_0 + \ww^\T\xx_i))^2 + \lambda ||\ww||_2^2
	\]
	with $\lambda = \sigma^2 / \tau^2$.
	\newline
	\item (\textbf{math}) Find a closed form solution $\xx^\star$ to the ridge regression
	problem:
	\[
	\text{minimize: } ||A\xx - \bb||_2^2 + ||\Gamma\xx||_2^2.
	\]
	
	\item
	(\textbf{implementation}) Attempt to predict the $\log\text{shares}$ using ridge
regression from the previous problem solution. Make sure you include a bias
term and \textit{don't regularize the bias term}.
Find the optimal regularization parameter $\lambda$
from the validation set. Plot both $\lambda$ versus the validation RMSE (you should have
tried at least 150 parameter settings randomly chosen between 0.0 and 150.0 because
the dataset is small)
and $\lambda$ versus $||\thetab^\star||_2$ where $\thetab$ is your weight vector.
What is the final RMSE on the test set with the optimal $\lambda^\star$?\\
\newline
(continued on the following pages)
\end{enumerate}
\end{problem}
\begin{solution}
	\vfill
\end{solution}
\newpage

\begin{problem}[3 (continued)]
\begin{enumerate}[(a)]
	\setcounter{enumi}{3}
\item (\textbf{math}) Consider regularized linear regression where we pull the
basis term out of the feature vectors. That is, instead of computing $\hat\yy
= \thetab^\T\xx$ with $\xx_0 = 1$, we compute $\hat\yy = \thetab^\T\xx + b$.
This corresponds to solving the optimization problem
\[
\text{minimize: } ||A\xx + b\1 - \yy||_2^2 + ||\Gamma\xx||_2^2.
\]
Solve for the optimal $\xx^\star$ explicitly. Use this close form to compute the
bias term for the previous problem (with the same regularization strategy). Make
sure it is the same.
\newline
\item (\textbf{implementation}) We can also compute the solution to the least squares
problem using gradient descent. Consider the same bias-relocated objective
\[
\text{minimize: } f = ||A\xx + b\1 - \yy||_2^2 + ||\Gamma\xx||_2^2.
\]
Compute the gradients and run gradient descent. Plot the $\ell_2$ norm
between the optimal $(\xx^\star, b^\star)$ vector you computed in closed form
and the iterates generated by gradient descent. Hint: your plot should move
down and to the left and approach zero as the number of iterations increases. If
it doesn't, try decreasing the learning rate.
\end{enumerate}
\end{problem}
\begin{solution}
	\begin{enumerate}[(a)]
		\item
		First, we substitute in the Gaussian's distribution and simplify.
		\begin{align*}
			&\argmax_{\ww} \sum_{i=1}^N \log \frac{1}{\sigma \sqrt{(2 \pi)}} 
				e^{(-\frac{(y_i - w_0 + \ww^\T\xx_i)^2}{2\sigma^2})} 
				+ \sum_{j=1}^D log \frac{1}{\tau \sqrt{2\pi}} 
				e^{(-\frac{w_j^2}{2 \tau^2})} \\
			=&\argmax_{\ww} \sum_{i=1}^N \log \frac{1}{\sigma \sqrt{(2 \pi)}} 
			- \frac{(y_i - w_0 + \ww^\T\xx_i)^2}{2\sigma^2} 
			+ \sum_{j=1}^D log \frac{1}{\tau \sqrt{2\pi}} 
			- \frac{w_j^2}{2 \tau^2} \\
			=&\argmax_{\ww} (N + D) \frac{1}{\sigma \sqrt{(2 \pi)}} 
			- \sum_{j=1}^N \frac{(y_i - w_0 + \ww^\T\xx_i)^2}{2\sigma^2} 
			- \sum_{j=1}^D \frac{w_j^2}{2 \tau^2}
		\end{align*}
		We now note that $N$, $D$, and $\sigma$ are constants and can be removed from the summation. Then, our expression is
		\begin{align*}
			&\argmin_{\ww}
			 \sum_{j=1}^N \frac{(y_i - w_0 + \ww^\T\xx_i)^2}{2\sigma^2} 
			+ \sum_{j=1}^D \frac{w_j^2}{2 \tau^2}\\
			=&\argmin_{\ww}
			 \sum_{j=1}^D (y_i - w_0 + \ww^\T\xx_i)^2
			+ \frac{2\sigma^2}{2 \tau^2} \sum_{j=1}^D w_j^2
		\end{align*}
		We let $\lambda = \frac{2\sigma^2}{2 \tau^2}$ so the expression becomes
		\begin{align*}
			&\argmin_{\ww}
			 \sum_{j=1}^N (y_i - w_0 + \ww^\T\xx_i)^2
			+ \lambda \sum_{j=1}^D w_j^2 \\
			=&\argmin_{\ww}
			\sum_{j=1}^N (y_i - w_0 + \ww^\T\xx_i)^2
		   + \lambda ||\ww||_2^2
		\end{align*}
		as desired.

		\item
		To minimize
		\[
			f = ||A\xx + b\1 - \yy||_2^2 + ||\Gamma\xx||_2^2,
		\]
		we take the gradient with respect to $\xx$ and set it to zero.
		\begin{align*}
			\nabla_{\xx} f &= \nabla_{\xx} [ (A\xx - \bb)^\T (A\xx - \bb) 
			+(\Gamma \xx)^\T (\Gamma \xx)] \\
			&= \nabla_{\xx} [ (\xx^\T A^\T - \bb^\T) (A\xx - \bb) 
			+(\xx^\T \Gamma^\T ) (\Gamma \xx) ]\\
			&= \nabla_{\xx} [ \xx^\T A^\T A \xx - 2\xx^\T A^\T \bb) 
			+ \bb^\T \bb + \xx^\T \Gamma^\T \Gamma \xx ]\\
			0 &= 2 A^\T A \xx - 2\T A^\T \bb) 
			+ 2 \Gamma^\T \Gamma \xx ]\\
			(A^\T A + \Gamma^\T \Gamma) \xx &= A^\T \bb \\
			\xx &= (A^\T A + \Gamma^\T \Gamma)^{-1} A^\T \bb \\
		\end{align*}
		Let $\Gamma = \sqrt{\lambda} \mathbf{I}$. Then,
		\[
			\xx = (A^\T A + \lambda \mathbf{I} )A^\T \bb, \\
		\]
		the solution to the minimization.
	\end{enumerate}
	\vfill
\end{solution}
\newpage

\end{document}
